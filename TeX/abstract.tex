%!TEX root = ../main.tex

\chapter*{Abstract} % (fold)
\label{cha:abstract}
\addcontentsline{toc}{chapter}{Abstract}

Process models are used in many model-based computational engineering activities, including process design, optimization, process control and simulation.
The construction of mathematical models is generally regarded as difficult and time consuming, and is therefore often handled by modelling experts.
If a tool were available that provided a systematic and safe approach for handling model complexity, consistent and correct models could be created and would eliminate the need for the expert. This project is part of an effort to further advance such tools.
% If a tool were available that provided a ways of handling model complexity in a systematic and safe approach for creating mathematical models.
% A process engineer could handle the modelling if a tool were available that provided the structure to the modelling process.

This thesis presents an information structure from which mathematical models can be extracted.
% This thesis presents a strategy for structuring information that can be extracted for representation of mathematical models. 
The hypothesis is that the information structure will facilitate the development of a structured modelling procedure that can be used in our computer-aided modelling tool.  % for the modelling of processes 
The information structure is in this thesis defined as the model ontology.
The ontology is based on a hierarchical decomposition of processes into model objects that can be described individually.
The first hierarchical level is a directed graph that consists of nodes and arcs.
% In the first hierarchical level are nodes and arcs. 
Nodes represent capacities that contain the fundamental extensive quantities such as mass, energy and information.
Arcs represent the transfer of extensive quantities between the nodes. 
The model objects are increasingly refined for each level in the hierarchical structure based on different characteristics such as phase and time scales. The modelling tool utilising the modelling ontology will then extract 
 building blocks from the hierarchical structure used for generating specific process models. 

The rules for generating the structure are implemented in a graphical editor, called the "Ontology Editor". 
The "Ontology Editor" is designed for constructing consistent ontologies for representation of mathematical models. 
The results gained from a case study are promising. The ontology allow for extraction of building blocks and variables that can be implemented in a our modelling tool that provide a systematic and safe approach for creating consistent and correct mathematical models.






% The main goal of this thesis was to explore an alternative method for structuring information in connection with modelling.
% The hypothesis is that this alternative method could lead to a simplification of the modelling procedure for modelling of dynamic processes in the field of physics and chemical engineering.
% Models of this kind are used to large extent in all kinds of engineering activities, such as process control, optimization and simulation.
% The modelling of physical-chemical processes is therefore an important task for a process engineer.
% The construction of these models is generally seen as a difficult and time consuming task and is often handled by modelling experts.
% This does not have to be the case if the modelling procedure is sufficiently simplified.

% The alternative method for structuring information uses a hierarchical decomposition of the process into objects that can be described individually.
% This decomposition result in two basic building blocks, namely nodes, which represent capacities that are able to contain the fundamental extensive quantities (such as mass and energy) and arcs, which describe the transfer of these fundamental extensive quantities through the common boundaries of the interconnected nodes. The new information structure uses different characteristics of the modelled process for identifying and sorting the nodes and arcs into defined types. When the type is identified, the model can be provided variables and equations for mathematical representation.

% Another goal for this thesis was the implementation of the information structure into an existing modelling tool, called the \PM. 
% This tool is designed to effectively assist a model designer with the construction of consistent process models and to reduce the needed time and overall effort.
% The results from the implementation are promising and allow definition of consistent process models within the framework of the defined information structure. The information structure is in this terminology called ontology.
% The alternative method for structuring information was implemented and tested in a prototype of a modelling tool. 


% Modelling is a difficult and time consuming task 
